% !TEX root = dbi.tex

%Advice for abstract

% The abstract must not contain references, as it may be used without the main
% article. It is acceptable, although not common, to identify work by author,
% abbreviation or RFC number. (For example, ``Our algorithm is based upon the
% work by Smith and Wesson.'')  Avoid use of ``in this paper'' in the
% abstract. What other paper would you be talking about here?  Avoid general
% motivation in the abstract. You do not have to justify the importance of the
% Internet or explain what QoS is.  Highlight not just the problem, but also the
% principal results. Many people read abstracts and then decide whether to
% bother with the rest of the paper.  Since the abstract will be used by search
% engines, be sure that terms that identify your work are found there. In
% particular, the name of any protocol or system developed and the general area
% (``quality of service'', ``protocol verification'', ``service creation
% environment'') should be contained in the abstract.  Avoid equations and
% math. Exceptions: Your paper proposes E = m c 2.

\begin{abstract}
  % 1) what is the DBI framework and why these are important? -- explain a bit
  % 2) what is the problem? 
  % 3) what did we do?
  % 4) and what is the result thus far?

  Dynamic binary instrumentation(DBI) frameworks enable the development of
  program analysis tools for unknown binary by facilitating automatic low-level
  instrumentation. This technique has become essential building blocks in
  developing tools that enhance the security and reliability of software
  systems. Representative DBI frameworks for these are PIN, Valgrind, and
  DynamoRIO.

  Forming a same layer between a running application and underlying operating
  system, each framework comes with different capabilities and performance
  implications as each framework's design principles varies up to their major
  target applications. However, developers who want to leverage the technology
  were rarely informed regarding how to choose a framework that meets their
  developmental requirements.

  In this talk, we present an ongoing project that compares three aforementioned
  frameworks by carefully examining three different aspects -- efficiency,
  capability, and usability. Purpose of our work is to provide a proper
  guideline to developers who want to develop security analysis tools enhancing
  this technology.
\end{abstract}

